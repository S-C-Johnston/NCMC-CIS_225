%&tex
Primitive types built in to java include:
\begin{description}
	\item [byte] 8 bit whole number (in most cases, sometimes this
		and other types is defined differently based on the
		``word length'' of the processor which handles these
		instructions. A 32 bit processor usually has a word
		length of 32, for instance, and a 64 bit processor often
		has a word length of 64).
	\item [short] 16 bit whole number
	\item [int] 32 bit whole number
	\item [long] 64 bit whole number
	\item [float] Floating point number capable of representing
		numbers with rational decimals. Often word length in
		bits, used frequently for high-performance decimal
		needs, such as rendering 3d graphics in games, though
		this is usually not the CPU performing those
		calculations.
	\item [double] A floating point number with double precision,
		for when you can't have the weirdness of representing
		decimals in binary get in the way of your accuracy.
	\item [char] A single character in Unicode 16-bit encoding.
	\item [boolean] A true or false value, which may possibly be
		represented as a single bit, but this is not specified
		in Appendix B of this book.
\end{description}
