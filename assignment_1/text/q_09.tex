%&tex
I chose to replicate the image of the house and the sun.

\begin{enumerate}
	\item Create the object instances and make them visible:
		\begin{enumerate}
			\item new Square() -> house
			\item new Square() -> window
			\item new Triangle() -> roof
			\item new Circle() -> sun
		\end{enumerate}
	\item Move (horizontally -150 pixels) and resize (150 pixels)
		house square. The default color is red, so changing the
		color isn't necessary.
	\item Recolor (``black'') and \em{then} move (horizontally -120
		pixels, invoked moveDown once), the window square.
	\item Triangle default color is green, so changing it is not
		necessary. Position is drawn from the center of the base
		of the triangle, and knowing this and the position of
		the other squares, the position can be set precisely.
		Move (vertically -80 pixels, horizontally 25 pixels) and
		resize (height 60, width 200) to line up with the top
		center of the house.
	\item Recolor (``yellow'') the sun circle and move it
		(horizontally 150 pixels, vertically -50 pixels).
\end{enumerate}

This process could presumably be done in different ways, not the least
of which are different orders of operations, as long as the black window
square is touched after the red house square is touched. Movements are
of course based on the default starting positions. If positions and
colors on the canvas can be given as a parameters when creating the
object, then this could be done more or less all in one shot during
object instantiation.
