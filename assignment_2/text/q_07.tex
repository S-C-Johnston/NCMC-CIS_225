%&tex
\begin{description}
	\item [class vs public first:] The order of the keywords does
		matter. The access modifier (public) always precedes
		everything else on a line that we've so far seen. Class
		immediately precedes an identifier for a class.
	\item [Change in class diagram:] when a file has errors which
		prevent it from compiling, the BlueJ class diagram
		paints over its icon with cross-hatches.
	\item [Error seen on change:] the error seen when this malformed
		change is made is that identifier was expected, rather
		than a keyword. Also, as a result, the class constructor
		shows an error that it requires a valid type signature,
		since the editor has no way of knowing that the
		constructor belongs to the malformed class declaration.
	\item [Does the error make sense:] the error reported only makes
		sense because I know that the class keyword requires an
		identifier as an immediate argument. If that was a
		foreign concept to me, it would appear cryptic, and even
		as such I'm not sure I'd be able to make immediate sense
		of it.
\end{description}
